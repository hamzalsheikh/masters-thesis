%%% -*-LaTeX-*-
%%% This is the abstract for the thesis.
%%% It is included in the top-level LaTeX file with
%%%
%%%    \preface    {abstract} {Abstract}
%%%
%%% The first argument is the basename of this file, and the
%%% second is the title for this page, which is thus not
%%% included here.
%%%
%%% The text of this file should be about 350 words or less.

% Summary from Proposal
%Computer clustering is the grouping of servers to increase scalability and
%offer a unit of orchestration. One crucial component of orchestration is
%scheduling, i.e. finding a suitable set of nodes in the cluster to run a job.
%In this work, we aim to leverage cross-cluster communication and build a
%system that allows clusters to trade resources and improve overall
%performance. We evaluate this by running simulated experiments to find
%workloads and policies that would benefit from such a system, and test various
%policies on available production workloads.

% overview and relevance
%The need for more compute power resulted in creating distributed systems
%distributed systems complexity demanded infrastructure systems to automate
%tasks, multi cluster environments are being used more than ever. Some reasons:
%scalability issues, geographical constraints, multi-provider strategies,
%orchestrate accross various locations, unify access to infrastructure, migrate
%apps from one cluster to the other, placement optimization, ...

%% pain point / motivate the problem
%Limitations: pod placements, resource fragmentation, unused resources,
%increased cost, ... Proposed solution To ease the limitations while still
%reaping the benefits, we propose CloudStreet, a per cluster trading mechanism
%allowing clusters to exchange resources between each other and optimize their
%performance with user specified policies. Results Our initial results run a
%simulator shows promising outcomes

%% Actual Start

Computer clusters were developed to overcome the limitations of scaling
applications on a single machine. With the widespread adoption of
containerization, driven by its benefits in consistency, portability, and
efficient resource utilization, container orchestration systems like
Kubernetes, Docker Swarm, and Nomad have become the standard for managing these
clusters. As orginzations' use cases expand and their reliance on orchestration
systems increases, managing the infrastructure with a single cluster has proven
insufficient. This led to growing demand for tools and systems capable of
managing multiple clusters, as multi-cluster environments are becoming
increasingly prevalent due to factors like scalability, geographical
considerations, multi-provider strategies, and workload partitioning. Despite
their promise, multi-cluster environments face various challenges, notably
resource fragmentation, increased cost, and reduced resource utilization. To
address these issues, we propose an intra-cluster resource trading mechanism.
Controlled by user defined rules and policies, the system facilitates
peer-to-peer resource exchange among clusters, allowing clusters to use
external resources when needed or offer their own to others when possible. We
evaluate the mechanism by running synthetic and production workloads on a
multi-cluster simulator. [WRITE RESULTS HERE]
