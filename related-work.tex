%%% -*-LaTeX-*-

\chapter{Related Work}

%Two types of related work 1- projects building multi cluster environments /
%supporting it
%
%Even a while back, google was exploring ways to support multi-vendor / cluster
%data stores
%https://storage.googleapis.com/gweb-research2023-media/pubtools/pdf/41216.pdf 
%
%Kubernetes created a special interest group focusing on solving challenges
%related to multi cluster environments. https://multicluster.sigs.k8s.io 
%
%hashicorp nomad offers multi-region federation for managing multiple clusters
%https://developer.hashicorp.com/nomad/tutorials/manage-clusters/federation
%
%% multi-cluster cloud native projects:
%
%These solutions offer a cluster-wide interconnection. 
%
%Skupper is a multi-cluster interconnection service that exposes namespaces to
%different clusters 
%
%Service meshes like istio and linkerd are also used to connect multiple
%clusters together. 
%
%2- papers building on multi cluster stuff (should i introduce lyra here or in
%the introduction? )
%
%- sky computing
%
%- liqo paper? - lyra?
%
%\blah
%
This section examines various approaches and solutions proposed and implemented
to address the complexities of multi-cluster environments.

Major cloud providers, Google Cloud Platform, Microsoft Azure, Amazon Web
Services, and Hashicorp's Nomad offer services for managing multi-cluster
environments. These tools provide various ways for exposing cluster resources
to each other, with a focus on load balancing between clusters and facilitating
service discovery in a multi-cluster setting. Our approach takes a different
route by concealing each cluster's workload and solely exposing resources
needs. This strategy allows clusters to dynamically expand and contract with
minimal exposure to the underlying workloads and services
\cite{noauthor_fleet_nodate, noauthor_azure_nodate,
noauthor_multi-region_nodate}.

In addition to cloud providers' offerings, multiple open source projects offer
tools for multi-cluster management. Cilium, Istio, and Linkerd are all service
meshes that offers multi-cluster connectivity. Those service meshes expose
gateways between clusters for communication, allowing services to communicate
through the clusters in the environment \cite{noauthor_cluster_nodate,
noauthor_multi-cluster_nodate-2, noauthor_install_nodate}. Submariner and
skupper are both early stage open source projects allowing direct multi
kubernetes cluster networking \cite{noauthor__nodate, noauthor_skupper_nodate}. 

These projects offer the network connectivity for multi-cluster environemnts,
exposing the services of the clusters to one other, without exposing the
control layer of these clusters. Our mechanism can be thought off as the
application layer on top of the network layer, introducing new sharing patterns
between the clusters.

Sky pilot offers a different perspective to managing multi-cluster
environments. It is an intercloud broker that was developed to run workloads on
different cloud providers to reduce costs and prevent provider lockup. Their
approach removes the need for cluster connectivity, relying on the
broker for efficient workload execution, whereas our approach allows each
cluster to make its own clustering decisions \cite{yang_skypilot_nodate}.

Lastly, Liqo is a kubernetes multi-cluster project that enables workload
offloading between clusters. It utilizes a peer-to-peer connection between
clusters and offers a static policy for allowing clusters to offload their
workloads on external clusters. Their focus is to create a continium of
resources across multiple clusters, exposing resources to one another. Our
mechanism have a very similar goal but uses trading as the basis of creating a
dynamic continium of resources when needed \cite{iorio_computing_2023,
noauthor_liqo_nodate}. 

In the next section, we introduce our mechanism and show how it fits in the
multi-cluster space.
