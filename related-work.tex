%%% -*-LaTeX-*-

\chapter{Related Work}

%Two types of related work 1- projects building multi cluster environments /
%supporting it
%
%Even a while back, google was exploring ways to support multi-vendor / cluster
%data stores
%https://storage.googleapis.com/gweb-research2023-media/pubtools/pdf/41216.pdf 
%
%Kubernetes created a special interest group focusing on solving challenges
%related to multi cluster environments. https://multicluster.sigs.k8s.io 
%
%hashicorp nomad offers multi-region federation for managing multiple clusters
%https://developer.hashicorp.com/nomad/tutorials/manage-clusters/federation
%
%% multi-cluster cloud native projects:
%
%These solutions offer a cluster-wide interconnection. 
%
%Skupper is a multi-cluster interconnection service that exposes namespaces to
%different clusters 
%
%Service meshes like istio and linkerd are also used to connect multiple
%clusters together. 
%
%2- papers building on multi cluster stuff (should i introduce lyra here or in
%the introduction? )
%
%- sky computing
%
%- liqo paper? - lyra?
%
%\blah
%
The three major cloud providers, GCP, Azure, and AWS offer services for
managing multi-cluster environments. However, they are limited to tools for
deployment and load balancing.   
%[CITE https://cloud.google.com/kubernetes-engine/docs/fleets-overview
%https://aws.amazon.com/blogs/containers/multi-cluster-management-for-kubernetes-with-cluster-api-and-argo-cd/
%https://azure.microsoft.com/en-us/products/kubernetes-fleet-manager#faq]
Hashicorp's Nomad offers first-class support of multi-cluster environments
through their multi-region federation service. However this exposes the
connected clusters' control plane, while allowing Jobs to run on all the
federated clusters. 
%[Cite
%https://developer.hashicorp.com/nomad/tutorials/manage-clusters/federation]
There are multiple open source projects building tools for multi-cluster
connectivity. Cilium, Istio, and Linkerd are all service meshes that offers
multi-cluster connectivity through various technologies.
% [CITE https://cilium.io/use-cases/cluster-mesh/
% https://linkerd.io/2.15/features/multicluster/
% https://istio.io/latest/docs/setup/install/multicluster/] 
Submariner and skupper are both early stage open source projects allowing
direct multi K8s cluster networking. 
%[CITE https://submariner.io  , https://skupper.io]

It is important to note that these projects offer the network connectivity for
multi-cluster environemnts, exposing the services of the clusters to one other,
without involving/exposing the control/scheduling layer of these clusters. We
later introduce cross-cluster scheduling capabilities steered by  actionable
policies to enhance the performance of multi-cluster environments.

Sky pilot is an intercloud broker that was developed to run workloads on
different cloud providers, reducing costs and preventing a provider lockup.
This papers offers a different point of view of how clussters can interact with
one another. 
%[CITE https://www.usenix.org/system/files/nsdi23-yang-zongheng.pdf]
Lastly, Liqo is a kubernetes multi-cluster project that enables workload
offloading between cllusters. It utilizes a peer-to-peer connection between
clusters and offers a static policy for allowing clusters to offload and accept
the offloading of foreign workload. The Liqo system does not infer
multi-cluster scheduling decisions from cluster state, they focus on creating a
static continium of resources accross clusters. Our work offers dynamic
scheduling between the clusters, allowing the clusters to benefit from
infromation they have about their cluster state. Moreover, a notion of
incentives is introduced to possibily encourage the exchange of resources
between the clusters.  
%[CITE liqo]

In the next section, we introduce our mechanism and show how it fits in the
multi-cluster space.
