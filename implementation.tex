%%% -*-LaTeX-*-

\chapter{Implementation} 

In the previous section, we outlined the theoretical design of the system. This
section goes over the implementation of the simulator written to evaluate the
mechanism.

%Why simulator 
We decided that having a simulator will allow us a broader scope for testing
the trading mechanism. Although it is possible to build the mechanism on top of
current multi-cluster networking solutions, we will be limited to the design
decisions of the tools used. Having a broader simulated results will open the
door for more interesting questions and answers leading to a concrete system
being built on an actual cluster.

The simulator is implemented as a set of seperate Go packages that can be run
in a distributed fashion.
\begin{enumerate}
  \item Trader: main module responsible for the functionality of the mechanism
  \item Scheduler: responsible for scheduling jobs on the cluster,
    communicating cluster state with the trader, providing/receiving virtual
    nodes 
  \item Client: Imports production traces or creates synthetic workloads and
    sends it to the schedueler.
\end{enumerate}

The simulator also includes other bookkeeping components like a service package
and a registry that don't interfere much with the evaluation and won't be
further discussed.

The client communicates with the scheduler through http, whereas as the
scheduler-trader and trader-trader communication is through grpc. 

\section{Scheduler} 

This section only discusses design decisioins and additions made relating to
the mechanism. These components are on top of regular scheduling functionality. 

The main aspect to be discussed in the scheduler is the scheduling algorithm
itself. It is a two level delay scheduling algorithm. A job is first added to
level 0 queue, and sent to level 1 queue if not scheduled in X time. Level 1
jobs are more likely to be scheduled away from their data, hence these are the
ones that will be sent to the trader when estimating the size of the node to
send. A distinctive difference to the delay scheduling algorithm is that the
nodes do not request jobs to be scheduled on them, but instead the scheduling
algorithm looks for nodes to schedule the jobs, similar to widely used cluster
schedulers.

A node burst scheduling is also implemented in the scheduler. Once a virtual
node is added to the cluster, the cluster tries to schedule as much viable jobs
it can on it right away to increase utilization of the virtual node, since the
node have a deadline. 

The scheduler also implements a gRPC server responsible for sending the
required cluster state to the trader.

\section{Trader}

The trader receives the cluster state from the scheduler through a continious
loop as a gRPC client. The trader then runs a state monitor, cycling through
the outgoing policies to check whether any is broken.   

\subsection{Trading algorithm}

When cluster state breaks a policy, trading is initaited. The trader retrieves
a list of active traders from the regisrty, estimates the amount of resources
needed while abiding to constraints, and sends a contract request to all
participating traders. It then chooses the most favorable trader, approves the
contract, and adds the received virtual node to the local cluster. 

\subsection{Policies}

\lstinputlisting[language=go, firstline=94,
lastline=96]{../multi-cluster-simulator/pkg/trader/trader.go}

The outgoing policies are implemented as an interface with a Broken method
signature. This allows the policies to be pluggablle. The user can define the
policy as long as there's a notion that the policy can break depending on the
cluster changes.

We currently implement two Policies:
\begin{enumerate}
  \item Wait time: The policy is broken if the average wait time in the cluster
    is above a threshold.
  \item Utilization: The policy is broekn if resource utilization exceeds a
    limit. 
\end{enumerate}

Incoming policies do not have the pluggibility trait. Users dial the knobs for
accepting incoming contracts with respect to the cluster state.
